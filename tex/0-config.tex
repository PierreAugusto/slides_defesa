% !TeX root = ../main.tex

\documentclass[t, compress, xcolor={table,dvipsnames}, aspectratio=169, 10pt]{beamer}
%\includeonlyframes{current}

\setbeamercovered{transparent} % Transparent pauses
% To change document language
\usepackage[portuguese]{babel}
% To include special characters directly from the keyboard (for portuguese language)
\usepackage[utf8]{inputenc}
% To include figures (not necessary in Beamer)
%\usepackage{graphicx}
% To include SVG images
\usepackage[notransparent]{svg}
% Hyperlinks
\usepackage{hyperref}
% To improve the positioning of figures
\usepackage{float}
% To adjust the page layout
\usepackage{geometry}
% To create better mathematical formulas
\usepackage{amsmath}
% To include more symbols in mathematical formulas
\usepackage{amssymb}
% To include links
\usepackage{url}
% To include PDF pages
\usepackage{pgfpages}
% % Apendix
% \usepackage{appendixnumberbeamer}
% To adjust the enumerate style
\usepackage{enumerate}
% To allow color changes in the text
\usepackage{color}
% To create improved tables
\usepackage{booktabs,multirow,threeparttable}
% To have the hability to include captions in non-floats
%\usepackage{capt-of}
\usepackage{subcaption}
\usepackage[font=scriptsize]{caption}
% Strikethrough text
\usepackage[normalem]{ulem}
% Coloured boxes
\usepackage{tcolorbox}
% % To include python codes
% \usepackage{minted}
% \usemintedstyle{monokai}
% \definecolor{MintedBG}{gray}{0.2}
% \definecolor{MintedNumbBG}{gray}{0.6}
% %https://tex.stackexchange.com/questions/173850/problem-in-adding-a-background-color-in-a-minted-environment
% \usepackage{tcolorbox}
% \tcbuselibrary{minted,skins}
% \newtcblisting{PyCode}[1][]{
%   listing engine=minted,
%   colback=MintedBG,
%   colframe=black!70,
%   listing only,
%   minted style=monokai,
%   minted language=python,
%   minted options={linenos=true,numbersep=3mm,texcl=true,#1},
%   left=5mm,enhanced,
%   overlay={\begin{tcbclipinterior}\fill[MintedNumbBG] (frame.south west)
%             rectangle ([xshift=5mm]frame.north west);\end{tcbclipinterior}}
% }
% To change the font of the slides
\usepackage{fontspec}
% The Inter font
\usepackage{inter}
% To define the citation style
\usepackage[style=authoryear]{biblatex}
%\addbibresource{TexFiles/Bibliography.bib}
% To include lipsum text
\usepackage{lipsum}
% To use the \num number formatter
\usepackage{siunitx}
% To justify slides
\usepackage{ragged2e}
\apptocmd{\frame}{}{\justifying}{}
% To reduce the space between ToC items
\usepackage{etoolbox}
\makeatletter
\patchcmd{\beamer@sectionintoc}{\vskip1.5em}{\vskip1.3em}{}{}
\makeatother
% Tikz to draw fluxograms
\usepackage{tikz}
\usetikzlibrary{shapes,arrows}
\usetikzlibrary{calc}
\usetikzlibrary{fadings}
\usetikzlibrary {arrows.meta,bending,positioning}

% Defining the dark theme (and other slide settings)
\definecolor{primary}{RGB}{219, 29 29} % Primary color
\definecolor{SPLUSRed}{RGB}{219, 29, 29} % First (start) color of progress bar gradient
\definecolor{SPLUSBlue}{RGB}{72, 33, 214} % Second (end) color of progress bar gradient
\definecolor{LightGray}{RGB}{179, 179, 179} % Color of the background of the progress bar and email+institute at bottom of slides

\usetheme[background=dark]{metropolis} % Dark Metropolis theme
\metroset{progressbar=foot} % Progress bar on page footer

\usecolortheme[named=primary]{structure}

% To make the tile, institute, date, author and subtitle boxes transparent
\setbeamercolor{title}{bg=}
\setbeamercolor{institute}{bg=}
\setbeamercolor{date}{bg=}
\setbeamercolor{author}{bg=}
\setbeamercolor{subtitle}{bg=}
% The color of the progress bar
\setbeamercolor{progress bar}{fg=primary}
% The color of the title (of the slides). Primary in the background, with white letters
\setbeamercolor{frametitle}{bg=, fg=white}
% The color of the text
\setbeamercolor{normal text}{fg=white}

% Setting the geometry of the slides (0.6 border on the left and right)
\geometry{left=0.6cm, right=0.6cm}

% To put a number on figures
\setbeamertemplate{caption}[numbered]

% This line sets the background of the slides (except the title slide)
\usebackgroundtemplate
{
    \includegraphics[width=\paperwidth,height=\paperheight]{images/Slides/Conteudo1_Azul.png}
    %\includegraphics[width=\paperwidth,height=\paperheight]{images/Slides/Conteúdo1_SemCor_Azul.png}
    %\includegraphics[width=\paperwidth,height=\paperheight]{images/Slides/Conteúdo2_Azul.png}
    %\includegraphics[width=\paperwidth,height=\paperheight]{images/Slides/Conteúdo2_SemCor_Azul.png}
}

%%%%%%%%%%%%%%%%%
% Other settings
\makeatletter

% The size of progress bar and separators of the Metropolis theme
\setlength{\metropolis@progressinheadfoot@linewidth}{4pt}
\setlength{\metropolis@titleseparator@linewidth}{3pt}
\setlength{\metropolis@progressonsectionpage@linewidth}{4pt}

% Reducing space between the frame and the frame title
%https://tex.stackexchange.com/questions/369090/reduce-space-between-mini-frame-and-frame-title
\setbeamertemplate{frametitle}{
  \nointerlineskip
  \begin{beamercolorbox}[
      wd=\paperwidth,
      sep=0pt,
      leftskip=.3cm,
      ht=.8cm,
      dp=0ex,
    ]{frametitle}
  \insertframetitle
  \nolinebreak
  \end{beamercolorbox}
  }

% Custom title page
\setbeamertemplate{title page}{
  \vbox{}
  \vfill
  \begingroup
  \vspace{3cm}
    \begin{minipage}{\textwidth}
    % This minipage contains the title and subtitle
    \begin{beamercolorbox}[sep=8pt, left]{title}
      \usebeamerfont{title}\inserttitle\par%
      \ifx\insertsubtitle\@empty%
      \else%
        \vskip0.25em%
        {\usebeamerfont{subtitle}\usebeamercolor[fg]{subtitle}\insertsubtitle\par}%
      \fi%     
    \end{beamercolorbox}%
    \end{minipage}

    \vskip0pt plus 1filll
    \noindent%
    \begin{minipage}{.65\textwidth}
    % This minipage has two columns. One (left) with the author, institute (long) and date, and the other (right) with the founding agency logo
    \begin{beamercolorbox}[sep=8pt, left]{infos}
      \usebeamerfont{author}\insertauthor\\
      \usebeamerfont{institute}\insertinstitute\\
      \usebeamerfont{date}\insertdate
    \end{beamercolorbox}
    \end{minipage}
    \hfill%
    \begin{minipage}{0.30\linewidth}
    {\hfill\includegraphics[width=0.7\linewidth]{images/Logo_CNPq.png}}
    \end{minipage}
    \vspace{0.5cm}
  \endgroup
}

% Color gradient of progress bars
% https://www.reddit.com/r/LaTeX/comments/j2t15h/gradient_in_metropolis_beamer_progress_bar/
\def\lenn{60pt}
\pgfdeclarehorizontalshading{SPLUS}{\paperheight}{
color(0pt)=(SPLUSRed); color(\lenn)=(SPLUSBlue)}

% Footer progress bar
\setbeamertemplate{progress bar in head/foot}{
    \nointerlineskip
    \setlength{\metropolis@progressinheadfoot}{%
    \paperwidth * \ratio{\insertframenumber pt}{\inserttotalframenumber pt}%
    }%
    \begin{beamercolorbox}[wd=\paperwidth]{progress bar in head/foot}
    \begin{tikzpicture}[shading=SPLUS]
        \fill[LightGray] (0,0) rectangle (\paperwidth, \metropolis@progressinheadfoot@linewidth);
        \shade (0,0) rectangle (\metropolis@progressinheadfoot, \metropolis@progressinheadfoot@linewidth);
    \end{tikzpicture}%
    \end{beamercolorbox}
}

% Section page progress bar
\setbeamertemplate{progress bar in section page}{
    \nointerlineskip
    \setlength{\metropolis@progressonsectionpage}{%
    0.5\paperwidth * \ratio{\insertframenumber pt}{\inserttotalframenumber pt}%
    }%
    \vspace{.5cm}
    \begin{beamercolorbox}[wd=0.5\paperwidth]{progress bar in section page}
    \begin{tikzpicture}[shading=SPLUS]
        \fill[LightGray] (0,0) rectangle (0.5\paperwidth, \metropolis@progressonsectionpage@linewidth);
        \shade (0,0) rectangle (\metropolis@progressonsectionpage, \metropolis@progressonsectionpage@linewidth);
    \end{tikzpicture}%
    \end{beamercolorbox}
}

% \makeatother
\DeclareSIUnit{\angstrom}{\textup{\AA}} % Angstrom unit, to be used with the siunitx package
% % End of other settings
% %%%%%%%%%%%%%%%%%%%%%%%

% % Remove bottom navigation symbols
% \setbeamertemplate{navigation symbols}{}

% Add the email and short institute to the bottom of the slide
\setbeamertemplate{frame footer}{\textcolor{LightGray}{\vspace{-0.2cm}\insertshortauthor~(\insertshortinstitute)}}

% Change Beamer bullets, adjust spacing and change color
\setbeamertemplate{itemize item}[circle]
\setbeamertemplate{itemize subitem}{\raisebox{0.08cm}{\tiny$\blacktriangleright$}}

% %\setbeamertemplate{itemize item}{\color{SPLUSRed}$\bullet$}
% %\setbeamertemplate{itemize subitem}{\color{SPLUSBlue}$\bullet$}

% Change spacing between items in the itemize environment
\let\olditemize=\itemize \let\endolditemize=\enditemize \renewenvironment{itemize}{\olditemize \itemsep0em}{\endolditemize}

%https://tex.stackexchange.com/questions/5941/changing-left-margin-in-itemize-environment-of-beamer-class
\settowidth{\leftmargini}{\usebeamertemplate{itemize item}}
\addtolength{\leftmargini}{2\labelsep}

% Defining custom fonts
%\newfontfamily\robotobold{Roboto Condensed Bold}
%\newfontfamily\inter{Inter}

\newfontfamily\robotobold[
  Path = ./fonts/,
  Extension = .ttf,
  UprightFont = *-Regular,
  BoldFont = *-Bold,
  ItalicFont = *-Italic
]{RobotoCondensed}
\newfontfamily\inter[
  Path = ./fonts/,
  Extension = .ttf,
  UprightFont = *-Regular,
  BoldFont = *-Bold,
  ItalicFont = *-Italic
]{Inter}

% Roboto for the title and subtitle
\setbeamerfont{title}{family=\robotobold, size={\fontsize{26}{24}}, series=\bfseries}
\setbeamerfont{subtitle}{family=\robotobold, size={\fontsize{18}{16}}, series=\bfseries}
\setbeamerfont{section title}{family=\robotobold, size={\fontsize{20}{18}}}
\setbeamerfont{frametitle}{family=\robotobold, size={\fontsize{17}{0}}}

% Inter for the other text
\setbeamerfont{author}{family=\inter}
\setbeamerfont{institute}{family=\inter}
\setbeamerfont{date}{family=\inter}
\setbeamerfont{normal text}{family=\inter}
\usefonttheme[onlymath]{serif}

\AtBeginDocument{\usebeamerfont{normal text}}

\let\tempone\itemize
\let\temptwo\enditemize
\renewenvironment{itemize}{\tempone\addtolength{\itemsep}{0.5\baselineskip}}{\temptwo}

% Color boxes
% https://tex.stackexchange.com/questions/66154/how-to-construct-a-coloured-box-with-rounded-corners
\tcbset{colback=SPLUSBlue!10!black, 
        colframe=SPLUSBlue!40!black, 
        coltext=white,
        left=1mm,
        right=1mm}

\newtcolorbox{splusbox}[1]{colback=SPLUSBlue!10!black, colframe=SPLUSBlue!40!black, 
                           coltext=white,
                           left=1mm,
                           right=1mm,
                           lefttitle=0.1mm,
                           title=#1}

\newtcolorbox{splusbox2}[1]{colback=SPLUSBlue!10!black, colframe=SPLUSBlue!40!black, 
                           coltext=white,
                           left=0.5cm,
                           lefttitle=0.1mm,
                           title=#1}

\setlength\parskip{\baselineskip} % for example

% https://tex.stackexchange.com/questions/55589/text-justify-in-beamer
\apptocmd{\frame}{}{\justifying}{} % Allow optional arguments after frame.
\apptocmd{\column}{}{\justifying}{} % Allow optional arguments after frame.
\apptocmd{\itemize}{}{\justifying}{} % Allow optional arguments after frame.

% Ending slide
% https://tex.stackexchange.com/a/338461
\setbeamertemplate{endpage}{%
\usebackgroundtemplate{
    \includegraphics[width=\paperwidth,height=\paperheight]{images/Slides/Capa3_Azul.png}
    }
      \begin{frame}[c]
        \centering
        \vspace{1cm}
        {\Huge \textbf{Obrigado!}}
      \end{frame}

  \usebackgroundtemplate{
      \includegraphics[width=\paperwidth,height=\paperheight]{images/Slides/Conteúdo1_Azul.png}
    }
  }
%\usebeamertemplate{endpage}

% Adjust itemize left margin
\settowidth{\leftmargini}{\usebeamertemplate{itemize item}}
\addtolength{\leftmargini}{\labelsep}